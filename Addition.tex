\documentclass[12pt]{article}

% Packages
\usepackage[margin=1in]{geometry}
\usepackage{amsmath, amssymb, amsthm}
\usepackage{enumitem} % Better control over lists
\usepackage{fancyhdr} % Header/Footer
\usepackage{lastpage} % For total page count in footer
\usepackage{hyperref} % Clickable refs and links
\usepackage{tikz}     % For diagrams (optional)

% Header/Footer formatting
\pagestyle{fancy}
\fancyhf{}
\rhead{}
\cfoot{\thepage\ of \pageref{LastPage}}

% Custom commands
\newcommand{\points}[1]{\hfill {[#1 points]}}
\newcommand{\problem}[2][]{%
  \item {#2}%
  \ifx&#1&%
  \else%
    \points{#1}%
  \fi
  \par\vspace{0.5em}
}

% Title info
\title{\vspace{-1cm}\textbf{Building Calculus From Scratch }}
\author{With Terence Tao's \textit{Analysis I}}

\begin{document}

\maketitle
\vspace{-1em}

\newenvironment{justifiedproof}{%
  \begin{tabular}{@{}p{0.6\linewidth}@{\quad}p{0.12\linewidth}@{}}%
}{\end{tabular}}

\section{Starting Axioms}
\begin{enumerate}
	\item $0$ is a natural number.
	\item if $n$ is a natural number, then its successor $n$++ is a natural number. We define the short-hand symbols
	$$1 := 0\text{++}$$
	$$2 := (0\text{++})\text{++}$$
	$$3 := ((0\text{++})\text{++})\text{++}$$
	$$\hdots$$
	\item $0$ is not the successor of any natural number. That is, $0 \neq n\text{++}$ for any $n$.
	\item Different natural numbers have different successors. That is, $[n \neq m] \iff [n\text{++} \neq m\text{++}]$ for any $n,m$. Equivalently, if we have that $n\text{++} = m\text{++}$, then we have that $n=m$.
	\item The principle of induction holds. That is, let $P(n)$ be a property of the natural number $n$. Let $P(0)$ be true, and let $P(n) \implies P(n\text{++})$. Then $P(n)$ is true for all natural numbers $n$.
 \end{enumerate}


\section{Steps \& Problems}

\begin{enumerate}[leftmargin=*, label=\textbf{\arabic*.}]

    \problem{Define the addition operation on two natural numbers, $n$ and $m$.}
	We define incrementing $m$ by zero as $0 + m := m$. Suppose inductively, that we know how to increment $m$ by $n$. Then we can increment $m$ by $n$++ by defining $$(n\text{++}) + m := (n + m)\text{++}$$ This recursive definition allows us to now add numbers (perform repeated incrementation). For example, $2+m = (1\text{++})+m = (1+m)\text{++} = ((0\text{++})+m)\text{++} = ((0+m)\text{++})\text{++} = ((m)\text{++})\text{++}$, which is exactly $m$ incremented twice.
	
    \problem{Prove that $n+0 = n$ for any natural number $n$}
        	\textcolor{blue}{ We will induct on $n$. We have that $0+0 = 0$ by the fact that $0 + m := m$ for any natural number $m$, including $0$. Suppose, inductively, that $k+0=k$ for some natural number $k$. Now consider the sum $(k\text{++}) + 0 = (k+0)\text{++} \overset{\text{I.H.}}{=} (k)\text{++}$. Hence for any natural number $n$, we have that $n+0 = n$.
}

    \problem{Prove that $n + (m\text{++}) = (n + m)\text{++}$}
        	\textcolor{blue}{We will induct on $n$. For $n = 0$, we have that $0 + (m\text{++}) = m\text{++} = (0 + m)\text{++}$. Assume inductively that $k + (m\text{++}) = (k + m)\text{++}$ for some natural number $k$. Now for the next number, $k\text{++}$, we have that $(k\text{++}) + (m\text{++}) = (k + (m\text{++}))\text{++}$ (by the definition from problem 1) $\overset{\text{I.H.}}{=} ((k+m)\text{++})\text{++} = ((k\text{++}) + m)\text{++}$. Hence, $n + (m\text{++}) = (n + m)\text{++}$ for all natural numbers $n$.}
	
	
	 \problem{Prove that the natural numbers are closed under addition. That is, if $a$,$b$ are natural numbers, then $a+b$ is a natural number.}
    	\textcolor{blue}{We will fix $b$ and induct on $a$. Base case: $0$ is a natural number by Axiom $1$ and let $b$ be a natural number. Then $0+b = b$ by our definition of incrementing by zero. But $b$ is a natural number (by assumption). This closes the base case. Assume inductively that for some natural number $a$, if $a$ and $b$ are natural numbers then so is $a+b$. Consider the next natural number $a\text{++}$. By Axiom $2$, since $a$ is a natural number (I.H), we must have that $a\text{++}$ is also a natural number. Now consider the sum $(a\text{++}) + b = (a+b)\text{++}$. We have that $a$ and $b$ are natural numbers, so $a+b$ is a natural number by the inductive hypothesis. By Axiom $2$, we have that its successor $(a+b)\text{++}$ must also be a natural number. This closes the induction. Hence the natural numbers are closed under addition.}
	
	 \problem{Prove that the addition operation is commutative. That is, for any two natural numbers, $a$ and $b$, we have that $a+b = b+a$}
    	\textcolor{blue}{We will fix $b$ and induct on $a$. We have that $0 + b = b$ from our first definition—incrementing a natural number by 0. Then $b = b + 0$, which we have from problem $2$'s proof. Hence we have our base case $0 + b = b + 0$. Suppose inductively that $a + b = b + a$ for some natural number $a$. Now consider the next number, $a\text{++}$ : we have $(a\text{++}) + b = (a + b)\text{++} \overset{\text{I.H.}}{=} (b + a)\text{++} = b+(a\text{++})$ by our previous proof. Hence for some natural number $b$, we have that $a + b = b + a$ for all natural numbers $a$.}
	
	
	\problem{Prove that the additive cancellation law holds. That is, for any three natural numbers $a$, $b$, and $c$ such that $a + b = a + c$, we have $b = c$.}
    	\textcolor{blue}{We will fix $b$,$c$ and induct on $a$. For the base case $a=0$, we will show that $[0+b=0+c] \implies [c=b]$. Let us assume that $0+b=0+c$. We have that $0+b=b$ and that $0+c=c$ from a previous proof. Then by transitive equality, we have $b=c$. This closes the base case. Assume inductively that for some natural number $a$, we have that $[a+b=a+c] \implies [b=c]$. Now consider the next natural number $a\text{++}$. Assume that $(a\text{++})+b=(a\text{++})+c$. Re-writing both sides, we have $(a+b)\text{++} = (a+c)\text{++}$. $a+b$ is a natural number and $a+c$ is a natural number (by our additive closure proof). Recall by Axiom $4$ that if the successors of two natural numbers are equal, then the natural numbers themselves must be equal. Hence we have that $a+b = a+c$. By the inductive hypothesis ($[a+b=a+c]  \implies [b=c]$), we must now have that $b=c$. This closes the inductive step. Thus, for any $a$, $b$, $c$ such that $a + b = a + c$, we also have that $b = c$.}
	
	\problem{A natural number $a$ is said to be positive iff $a \neq 0$. Prove that the sum of a natural number and a positive number must be positive.}
	\textcolor{blue}{Fix a positive number $b$. We will show that $a+b$ is positive for all $a$. Base case: $0 + b = b$ by our definition of incrementing by zero, and $b$ is positive (by assumption). This closes the base case. Assume inductively that $a+b$ is positive for some natural number $a$. Then consider the statement for the next natural number $a\text{++}$: we have $(a\text{++}) + b = (a+b)\text{++}$. By the inductive hypothesis, we have that $a+b$ is positive. But this means that $(a+b)\text{++}$, the succesor of $a+b$, must also be positive—it cannot be $0$ because of Axiom $3$: zero is not the successor of any natural number. This closes the induction.}
	
	\problem{Prove that, for natural numbers $a$ and $b$, if $a + b = 0$ then $a = 0$ and $b = 0$.}
	\textcolor{blue}{We will prove the equivalent contraposition statement: if $a \neq 0$ or $b \neq 0$, then $a + b \neq 0$. Assume that either $a \neq 0$ or $b \neq 0$. In the case that $a \neq 0$, $a+b$ must be positive (according to the result of the previous proof) and hence non-zero. In the case that $b \neq 0$, $a+b$ must similarly be positive and hence non-zero.}
	
    \problem{(\textbf{EXERCISE 1}) Prove that the addition operation is associative. That is, for any three natural numbers, $a$, $b$, $c$, we have that $(a+b)+c = a+(b+c)$}
    	\textcolor{blue}{We will fix $b,c \in \mathbb{N}$ and induct on $a$. Base case:
	\begin{align*}
	(0 + b) + c &=b + c \\
	&= 0 + (b + c)\\
	\end{align*}
	Assume inductively that $(a + b) + c = a + (b + c)$ for some natural number $a$. Then
	\begin{align*}
	((a\text{++}) + b) + c &= ((a + b)\text{++}) + c\\
	&= ((a+b) + c)\text{++}  \\
	& = (a + (b + c))\text{++} \;\; \text{(I.H)} \\
	&= (a\text{++}) + (b+c) \\
	\end{align*}
	This closes the induction.}
	\\ \\ \\
	\problem{(\textbf{EXERCISE 2}) Let $a$ be a positive number. Prove that there exists exactly one natural number $b$ such that $b\text{++} = a$}
	\textcolor{blue}{ 
	We effectively need to prove that each positive $a$ has a unique predecessor $b$. Hence, it suffices to show that each $a$ has both: 1) \textit{at least} $1$ predecessor and 2) \textit{at most} $1$ predecessor. If $a$ is positive, then $a \neq 0$, which guarantees the existence of a $b$ such that $b\text{++} = a$ (a predecessor of $a$). Axiom $4$ (different natural numbers have different successors) guarantees that no two distinct natural numbers have the same successor. Hence, $a$ cannot have more than $1$ predecessor $\implies b$ must be the only predecessor of $a$.  \\ \\ \\
	Will will induct on $a$. Base case: for $a = 1$, we have $b=0 \implies b\text{++} = a$. This is the \textit{only} such $b$ for $a=0$ because a different $b \neq 0$ would have a different successor $b\text{++} \neq 1$ acccording to Axiom $4$: different natural numbers have different successors. Suppose inductively that for some natural number $a$ there exists a $b$ such that $b\text{++} = a$. Now consider the next natural number $a\text{++}$: the number $a\text{++}$, by definition, is the successor of $a$. So we have found one $b=a$ such that $b\text{++} = a\text{++}$. This is the only such $b$ because, similarly, by Axiom $4$, any other $b \neq a$ has a different successor $b\text{++} \neq a\text{++}$. This closes the induction.}
	
	\problem{Define the ordering of the natural numbers.}
	Let $n$ and $m$ be natural numbers. We say that $n$ is \textit{greater than or equal to} $m$, and write $n \geq m$ or $m \leq n$, iff we have $n = m + a$ for some natural number $a$. We say that $n$ is \textit{strictly greater than} $m$, and write $n > m$ or $m < n$, iff $n \geq m$ and $n \neq m$.
	
	\problem{(\textbf{EXERCISE 3}) Prove properties of order.}
	\begin{enumerate}
	
	   \problem{(Reflexivity) Prove that $a \geq a$ for any natural number $a$.}
    	\textcolor{blue}{
	\begin{align*}
	\text{For any natural $a$, we can find an $m$ such that $a = a + m$ ($a \geq a$) $\implies$ choose $m = 0$}
	\end{align*}}
	
	   \problem{(Transitivity) Prove that if $a \geq b$ and $b \geq c$, then $a \geq c$.}
    	\textcolor{blue}{
	\begin{align*}
	a \geq b, b \geq c \implies& a = b + k_0, b = c + k_1 \;\; \text{for some $k_0,k_1 \in \mathbb{N}$}\\
	\implies& a = (c + k_1) + k_0\\
	\implies& a = c + (k_1 + k_0) \;\; \text{(associativity)}\\
	\implies& a = c + m \;\; \text{for some natural $m = k_0 + k_1$} \;\; \text{(additive closure)}\\
	\implies& a \geq c
	\end{align*}}
	
	\problem{(Anti-symmetry) Prove that if $a \geq b$ and $b \geq a$, then $a = b$}
	\textcolor{blue}{
	\begin{align*}
	a \geq b, b \geq a \implies& a = b + k_0, b = a + k_1 \;\; \text{for some $k_0,k_1 \in \mathbb{N}$}\\
	\implies& a = (a + k_1) + k_0\\
	\implies& a = a + (k_1 + k_0) \;\; \text{(associativity)}\\
	\implies& a + 0 = a + (k_1 + k_0)\\
	\implies& 0 = k_1 + k_0 \;\; \text{(cancellation law)}\\
	\implies& k_0 = 0 \;\; \text{and} \;\; k_1 = 0 \;\; \text{(zero-sum proof)}\\
	\implies& a = b\\
 	\end{align*}}
	
	\problem{(Order preservation under addition) Prove that $a \geq b$ if and only if $a+c \geq b+c$.}
	\textcolor{blue}{
	\begin{align*}
	a \geq b \iff& \exists k (a = b+k)  \\
	\iff& a + c = (b + k) + c \\
	\iff& a + c = b + (k + c) \\
	\iff& a + c = b + (c + k) \\
	\iff& a + c = (b + c) + k \\
	\iff& a + c \geq b + c \\
	\end{align*}
	}
		\problem{Prove that $a < b$ if and only if $a\text{++} \leq b$.}
		\textcolor{blue}{
		\begin{align*}
	a < b \iff& \exists k (b = a+k \;\; \text{and} \;\; a \neq b)  \\
	\iff& a + k \neq a\\
	\iff& k \neq 0\\
	\iff& b = a + m\text{++} \;\; \text{with $k = m\text{++}$}\\
	\iff& b = a + (m+1)\\	
	\iff& b = a + (1+m)\\
	\iff& b = (a + 1)+m\\
	\iff& b = (a\text{++}) + m\\	
	\iff& a\text{++} \leq b\\	
	\end{align*}}
		
		\problem{Prove that $a < b$ if and only if $b = a + d$ for some positive number $d$.}
		\textcolor{blue}{
		\begin{align*}
	a < b \iff& \exists k (b = a+k \;\; \text{and} \;\; a \neq b)  \\
	\iff& a + k \neq a\\
	\iff& k \neq 0\\
	\iff& \text{choose} \;\; d = k\\
	\iff& b = a + d \;\; \text{and $d$ is positive}\\
	\end{align*}}
	
	\end{enumerate}
	
	\problem{\textbf{EXERCISE 4} Justify the following:}
	\begin{enumerate}
	\problem{$0 \leq b$ for all $b$.}	
	\textcolor{blue}{
	\begin{align*}
	\text{For any $b$, we can find an $m$ such that $b = 0 + m$ ($0 \leq b$)}
	\implies& \text{choose $m = b$}\\
	\end{align*}}
	
	\problem{If $a > b$, then $a\text{++} > b$}	
	\textcolor{blue}{
	\begin{align*}
	a > b \implies& a\text{++} > a > b \;\; \text{(definition of successor)}\\
	\implies& a\text{++} > b \;\; \text{(transitivity)}\\
	\end{align*}}
	
	\problem{If $a = b$, then $a\text{++} > b$}	
	\textcolor{blue}{
	\begin{align*}
	a = b \implies& a\text{++} > a = b \;\; \text{(definition of successor)}\\
	\implies& a\text{++} > b\\
	\end{align*}}
	\end{enumerate}
	
	\problem{\textbf{EXERCISE 5} Prove the principle of strong induction.}
	\textcolor{blue}{
	\begin{align*}
	x \implies& x \\
	\end{align*}}
	
\end{enumerate}

\end{document}
