\documentclass[12pt]{article}

% Packages
\usepackage[margin=1in]{geometry}
\usepackage{amsmath, amssymb, amsthm}
\usepackage{enumitem} % Better control over lists
\usepackage{fancyhdr} % Header/Footer
\usepackage{lastpage} % For total page count in footer
\usepackage{hyperref} % Clickable refs and links
\usepackage{tikz}     % For diagrams (optional)

% Header/Footer formatting
\pagestyle{fancy}
\fancyhf{}
\rhead{}
\cfoot{\thepage\ of \pageref{LastPage}}

% Custom commands
\newcommand{\points}[1]{\hfill {[#1 points]}}
\newcommand{\problem}[2][]{%
  \item {#2}%
  \ifx&#1&%
  \else%
    \points{#1}%
  \fi
  \par\vspace{0.5em}
}

% Title info
\title{\vspace{-1cm}\textbf{Addition and Multiplication HW}}
\author{\textit{Analysis I (TT)}}

\begin{document}

\maketitle
\vspace{-1em}

\newenvironment{justifiedproof}{%
  \begin{tabular}{@{}p{0.6\linewidth}@{\quad}p{0.12\linewidth}@{}}%
}{\end{tabular}}

\section{Starting Axioms}
\begin{enumerate}
	\item $0$ is a natural number.
	\item if $n$ is a natural number, then its successor $n$++ is a natural number. We define the short-hand symbols
	$$1 := 0\text{++}$$
	$$2 := (0\text{++})\text{++}$$
	$$3 := ((0\text{++})\text{++})\text{++}$$
	$$\hdots$$
	\item $0$ is not the successor of any natural number. That is, $0 \neq n\text{++}$ for any $n$.
	\item Different natural numbers have different successors. That is, $[n \neq m] \iff [n\text{++} \neq m\text{++}]$ for any $n,m$. Equivalently, if we have that $n\text{++} = m\text{++}$, then we have that $n=m$.
	\item The principle of induction holds. That is, let $P(n)$ be a property of the natural number $n$. Let $P(0)$ be true, and let $P(n) \implies P(n\text{++})$. Then $P(n)$ is true for all natural numbers $n$.
 \end{enumerate}


\section{Addition}

\begin{enumerate}[leftmargin=*, label=\textbf{\arabic*.}]

    \problem{Define the addition operation on two natural numbers, $n$ and $m$.}
	To begin, we must first define what adding two numbers means. We currently only have an increment operation, so we must define addition in terms of the increment operation—that is, we will \textit{define} addition as repeated incrementation. We define incrementing $m$ by zero as $0 + m := m$. Suppose inductively, that we know how to increment $m$ by $n$. Then we can increment $m$ by $n$++ by defining $$(n\text{++}) + m := (n + m)\text{++}$$ This recursive definition allows us to now add numbers (perform repeated incrementation). For example, $2+m = (1\text{++})+m = (1+m)\text{++} = ((0\text{++})+m)\text{++} = ((0+m)\text{++})\text{++} = ((m)\text{++})\text{++}$, which is exactly $m$ incremented twice.
	
    \problem{Prove that $n+0 = n$ for any natural number $n$}
        	\textcolor{blue}{ We will induct on $n$. We have that $0+0 = 0$ from $0 + m := m$ for any natural $m$, including $0$. Suppose, inductively, that $k+0=k$ for some natural number $k$. Then 
	\begin{align*}
	(k\text{++}) + 0 &= (k+0)\text{++} \;\; \text{(definition of addition)} \\
	&= (k)\text{++} \;\; \text{(I.H)} \\
	\end{align*}
	Hence for any natural number $n$, we have that $n+0 = n$.
}

    \problem{Prove that $n + (m\text{++}) = (n + m)\text{++}$}
        	\textcolor{blue}{We will induct on $n$. For $n = 0$,
	\begin{align*}
	0 + (m\text{++}) &= m\text{++} \\
	&= (0 + m)\text{++} \\
	\end{align*}
	Assume inductively that $k + (m\text{++}) = (k + m)\text{++}$ for some natural number $k$. Then
	\begin{align*}
	(k\text{++}) + (m\text{++}) &= (k + (m\text{++}))\text{++}  \;\; \text{(definition of addition)}\\
	&= ((k+m)\text{++})\text{++} \;\; \text{(I.H)} \\
	&= ((k\text{++}) + m)\text{++} \\
	\end{align*}
	Hence, $n + (m\text{++}) = (n + m)\text{++}$ for all natural numbers $n$.}
	
	
	 \problem{Prove that the natural numbers are closed under addition. That is, if $a$,$b$ are natural numbers, then $a+b$ is a natural number.}
    	\textcolor{blue}{Fix a natural number $b$. For $a = 0$, we have $a + b = b$ (natural). Assume inductively that $a+b$ is natural for some natural $a$. Then $(a\text{++}) + b = (a+b)\text{++}$ where $a+b$ is natural by the inductive hypothesis. Hence $(a+b)\text{++}$ is natural by Axiom $2$ (if a number is natural then its successor must also be natural). Hence the natural numbers are closed under addition.}
	
	 \problem{Prove that the addition operation is commutative. That is, for any two natural numbers, $a$ and $b$, we have that $a+b = b+a$}
    	\textcolor{blue}{Fix a natural number $b$. For $a=0$, 
	\begin{align*}
	a + b &= b \\
	&= b + 0 \;\; \text{(from proof that $n+0=n$ for any $n$)} \\
	&= b + a \\
	\end{align*}
	Assume inductively that $a + b = b + a$ for some natural $a$. Then
	\begin{align*}
	(a\text{++}) + b &= (a + b)\text{++} \;\; \text{(definition of addition)} \\
	&= (b + a)\text{++} \;\; \text{(I.H)} \\
	&= b+(a\text{++}) \;\; \text{(alternate-order proof of addition)} \\
	\end{align*}
	Hence for any $a,b$, we have that $a + b = b + a$.}
	
	
	\problem{Prove that the additive cancellation law holds. That is, for any three natural numbers $a$, $b$, and $c$ such that $a + b = a + c$, we have $b = c$.}
    	\textcolor{blue}{Fix natural numbes $b$ and $c$. For $a=0$, 
	\begin{align*}
	a+b = b \;\; \text{($0$-incrementing)} \\
	a+c = c \;\; \text{($0$-incrementing)} \\
	a+b=a+c \implies& b = c \;\; \text{(substitution)}\\
	\end{align*}
	Assume inductively that for some natural $a$, we have $[a+b=a+c] \implies [b=c]$. Then
	\begin{align*}
	(a\text{++})+b=(a\text{++})+c \implies& (a+b)\text{++} = (a+c)\text{++} \;\; \text{(definition of addition)} \\
	\implies& \text{$a + b$ and $a + c$ are naturals (additive closure)} \\
	\implies& \text{$a + b = a + c$ (Axiom $4$)} \\
	\implies& b = c \;\; \text{(I.H)} \\
	\end{align*}
	Hence, for any $a,b,c$, we have $[a + b = a + c] \implies [b = c]$}
	
	\problem{A natural number $a$ is said to be positive iff $a \neq 0$. Prove that the sum of a natural number and a positive number must be positive.}
	\textcolor{blue}{Fix a positive number $b$. Let $a = 0 \implies a + b = b \neq 0$. Then $(a\text{++}) + b = (a+b)\text{++} \neq 0$ since $a + b$ is a natural number (by our proof of additive closure), and the successor of any natural number is positive by Axiom $3$ (zero is not the successor of any natural number). Hence $a + b \neq 0$ for all $a$.}
	
	\problem{Prove that, for natural numbers $a$ and $b$, if $a + b = 0$ then $a = 0$ and $b = 0$.}
	\textcolor{blue}{We will prove the contrapositive:
	\begin{align*}
	\text{Let $a \neq 0$ or $b \neq 0$} \implies& \\
	& \text{if $a \neq 0$} \implies a+b \neq 0 \;\; \text{(by previous proof)} \\
	& \text{if $b \neq 0$} \implies a+b \neq 0 \;\; \text{(by previous proof)} \\
	\end{align*}
	Hence $a \neq 0$ or $b \neq 0 \implies a+b \neq 0$. Equivalently, \\ $a+b = 0 \implies a = 0$ and $b = 0$.}
	
    \problem{(\textbf{EXERCISE 1}) Prove that the addition operation is associative. That is, for any three natural numbers, $a$, $b$, $c$, we have that $(a+b)+c = a+(b+c)$}
    	\textcolor{blue}{We will fix $b,c \in \mathbb{N}$ and induct on $a$. Base case:
	\begin{align*}
	(0 + b) + c &=b + c \\
	&= 0 + (b + c)\\
	\end{align*}
	Assume inductively that $(a + b) + c = a + (b + c)$ for some natural number $a$. Then
	\begin{align*}
	((a\text{++}) + b) + c &= ((a + b)\text{++}) + c\\
	&= ((a+b) + c)\text{++}  \\
	& = (a + (b + c))\text{++} \;\; \text{(I.H)} \\
	&= (a\text{++}) + (b+c) \\
	\end{align*}
	This closes the induction.}
	\\ \\ \\
	\problem{(\textbf{EXERCISE 2}) Let $a$ be a positive number. Prove that there exists exactly one natural number $b$ such that $b\text{++} = a$}
	\textcolor{blue}{ 
	For the first positive number $a = 1$, choose $b = 0$ and we have $b\text{++} = a$. Then for the next natural $a\text{++}$, choose $b = a$ and we have $b\text{++} = a\text{++}$ (no induction hypothesis needed). Hence, every positive $a$ has at least $1$ predecessor $b$. Axiom $4$ (different natural numbers have different successors) guarantees that no two distinct natural numbers have the same successor. Hence, for every $a$, the corresponding $b$ that satisfies $b\text{++} = a$ must be the \textit{only} such $b$. \\ \\}
	
	\problem{Define the ordering of the natural numbers.}
	Let $n$ and $m$ be natural numbers. We say that $n$ is \textit{greater than or equal to} $m$, and write $n \geq m$ or $m \leq n$, iff we have $n = m + a$ for some natural number $a$. We say that $n$ is \textit{strictly greater than} $m$, and write $n > m$ or $m < n$, iff $n \geq m$ and $n \neq m$.
	
	\problem{(\textbf{EXERCISE 3}) Prove properties of order.}
	\begin{enumerate}
	
	   \problem{(Reflexivity) Prove that $a \geq a$ for any natural number $a$.}
    	\textcolor{blue}{
	\begin{align*}
	\text{For any natural $a$, we can find an $m$ such that $a = a + m$ ($a \geq a$) $\implies$ choose $m = 0$}
	\end{align*}}
	
	   \problem{(Transitivity) Prove that if $a \geq b$ and $b \geq c$, then $a \geq c$.}
    	\textcolor{blue}{
	\begin{align*}
	a \geq b, b \geq c \implies& a = b + k_0, b = c + k_1 \;\; \text{for some $k_0,k_1 \in \mathbb{N}$}\\
	\implies& a = (c + k_1) + k_0\\
	\implies& a = c + (k_1 + k_0) \;\; \text{(associativity)}\\
	\implies& a = c + m \;\; \text{for some natural $m = k_0 + k_1$} \;\; \text{(additive closure)}\\
	\implies& a \geq c
	\end{align*}}
	
	\problem{(Anti-symmetry) Prove that if $a \geq b$ and $b \geq a$, then $a = b$}
	\textcolor{blue}{
	\begin{align*}
	a \geq b, b \geq a \implies& a = b + k_0, b = a + k_1 \;\; \text{for some $k_0,k_1 \in \mathbb{N}$}\\
	\implies& a = (a + k_1) + k_0\\
	\implies& a = a + (k_1 + k_0) \;\; \text{(associativity)}\\
	\implies& a + 0 = a + (k_1 + k_0)\\
	\implies& 0 = k_1 + k_0 \;\; \text{(cancellation law)}\\
	\implies& k_0 = 0 \;\; \text{and} \;\; k_1 = 0 \;\; \text{(zero-sum proof)}\\
	\implies& a = b\\
 	\end{align*}}
	
	\problem{(Order preservation under addition) Prove that $a \geq b$ if and only if $a+c \geq b+c$.}
	\textcolor{blue}{
	\begin{align*}
	a \geq b \iff& \exists k (a = b+k)  \\
	\iff& a + c = (b + k) + c \\
	\iff& a + c = b + (k + c) \\
	\iff& a + c = b + (c + k) \\
	\iff& a + c = (b + c) + k \\
	\iff& a + c \geq b + c \\
	\end{align*}
	}
		\problem{Prove that $a < b$ if and only if $a\text{++} \leq b$.}
		\textcolor{blue}{
		\begin{align*}
	a < b \iff& \exists k (b = a+k \;\; \text{and} \;\; a \neq b)  \\
	\iff& a + k \neq a\\
	\iff& k \neq 0\\
	\iff& b = a + m\text{++} \;\; \text{with $k = m\text{++}$}\\
	\iff& b = a + (m+1)\\	
	\iff& b = a + (1+m)\\
	\iff& b = (a + 1)+m\\
	\iff& b = (a\text{++}) + m\\	
	\iff& a\text{++} \leq b\\	
	\end{align*}}
		
		\problem{Prove that $a < b$ if and only if $b = a + d$ for some positive number $d$.}
		\textcolor{blue}{
		\begin{align*}
	a < b \iff& \exists k (b = a+k \;\; \text{and} \;\; a \neq b)  \\
	\iff& a + k \neq a\\
	\iff& k \neq 0\\
	\iff& \text{choose} \;\; d = k\\
	\iff& b = a + d \;\; \text{and $d$ is positive}\\
	\end{align*}}
	
	\end{enumerate}
	
	\problem{\textbf{EXERCISE 4} Justify the following:}
	\begin{enumerate}
	\problem{$0 \leq b$ for all $b$.}	
	\textcolor{blue}{
	\begin{align*}
	\text{For any $b$, we can find an $m$ such that $b = 0 + m$ ($0 \leq b$)}
	\implies& \text{choose $m = b$}\\
	\end{align*}}
	
	\problem{If $a > b$, then $a\text{++} > b$}	
	\textcolor{blue}{
	\begin{align*}
	a > b \implies& a\text{++} > a > b \;\; \text{(definition of successor)}\\
	\implies& a\text{++} > b \;\; \text{(transitivity)}\\
	\end{align*}}
	
	\problem{If $a = b$, then $a\text{++} > b$}	
	\textcolor{blue}{
	\begin{align*}
	a = b \implies& a\text{++} > a = b \;\; \text{(definition of successor)}\\
	\implies& a\text{++} > b\\
	\end{align*}}
	\end{enumerate}
	
	\problem{\textbf{EXERCISE 5} (Prove the principle of strong induction). Let $m_0$ be a natural number and let $P(m)$ be a property of an arbitrary natural number $m$. Suppose that for each $m \geq m_0$, if $P(m')$ is true for all $m_0 \leq m' < m$, then $P(m)$ is also true. Prove that $P(m)$ is then true for all $m \geq m_0$.}
	\textcolor{blue}{
	We will prove the principle of strong induction by re-notating (and effectively reducing) the multi-variable inductive hypothesis to a single-variable inductive hypothesis. Then we will prove the strong induction principle by simply invoking the ordinary induction principle (Axiom $5$). \\ \\
	\textit{proof.} Suppose that for each $m \geq m_0$, if $P(m')$ is true for all $m_0 \leq m' < m$, then $P(m)$ is also true. Put formally, we are assuming: 
	\begin{equation}
	[\forall m \geq m_0(\forall m' (m_0 \leq m' < m) \implies P(m'))] \implies P(m)
	\label{eq:1}
	\end{equation}
	Let $Q(n)$ be the property that $P(m)$ is true for all $m_0 \leq m < n$. Note that $Q(n) \implies P(n)$ by \eqref{eq:1}. Hence, to show that $P(m)$ is true for all $m \geq m_0$, it suffices to show that $Q(n)$ holds for all $n \geq m_0$. \\ \\$Q(m_0)$ is true (vacuously). Suppose inductively that $Q(n)$ is true for some $n \geq m_0$ : that is, $P(m)$ is true for all $m_0 \leq m < n$. Then $P(n)$ is true by \eqref{eq:1}, meaning that $P(m)$ is true for all all of $m_0 \leq m < n+1$. But this is exactly the statement $Q(n+1)$. This closes the induction.}
	
	\problem{(\textbf{EXERCISE 6}) (Prove the principle of backwards induction). Let $n$ be a natural number and let $P(m)$ be a property pertaining to the natural numbers such that whenever $P(m\text{++})$ is true, then $P(m)$ is true. Suppose that $P(n)$ is true. Prove that $P(m)$ is true for all $m \leq n$.}
	\textcolor{blue}{The principle of backward induction starting on $n$ can be written as the statement
	\begin{equation}
	Q(n) := [P(n) \land \forall m (P(m\text{++}) \implies P(m))] \implies P(m) \forall m \leq n
	\label{eq:2}
	\end{equation}
	We effectively want to prove that $Q(n)$ is true for all $n$, and we can use ordinary forward induction to do this. We have that $$Q(0) : [P(0) \land \forall m (P(m\text{++}) \implies P(m))] \implies P(0)$$ is trivially true (as a tautology of the form $A \land B \implies A$). Suppose inductively that for some $n \geq 0$, $Q(n)$ is true. Now we will prove $Q(n+1)$: $$[P(n+1) \land \forall m (P(m\text{++}) \implies P(m))] \implies P(m) \forall m \leq n+1$$ \textit{proof:}
	\begin{align*}
	[P(n+1) \land \forall m (P(m\text{++}) \implies P(m))] \implies& P(n\text{++}) \\
	\implies& P(n) \\
	\implies& P(n) \land \forall m (P(m\text{++}) \implies P(m))  \;\; \text{(I.H)} \\
	\implies& P(m) \forall m \leq n \\
	\implies& P(m) \forall m \leq n \land P(n+1)  \\
	\implies& P(m) \forall m \leq n + 1\\
	\end{align*}}
	
	\problem{(\textbf{EXERCISE 7}) (Prove the principle of shifted induction) Let $n$ be a natural number and let $P(m)$ be a property pertaining to the natural numbers such that whenever $P(m)$ is true, then $P(m\text{++})$ is true. Show that if $P(n)$ is true, then $P(m)$ is true for all $m \geq n$.}
	
	\textcolor{blue}{	We want to show that $$Q(n) := [P(n) \land \forall m (P(m) \implies P(m\text{++}))] \implies P(m) \forall m \geq n$$ is true for all $n$. We can again achieve this using ordinary induction. The base case  $$Q(0) : [P(0) \land \forall m (P(m) \implies P(m\text{++}))] \implies P(m) \forall m \geq 0$$ is a direct re-stating of the ordinary principle of induction (Axiom $5$), and hence it must be true. Assume inductively that $Q(n)$ is true for some $n \geq 0$. Now we will prove $Q(n+1)$: 
	$$[P(n+1) \land \forall m (P(m) \implies P(m\text{++}))] \implies P(m) \forall m \geq n+1$$
	\textit{proof:} \\ \\
	Define $S(k) := P(n+1+k)$ for $k \in \mathbb{N}$.
	\begin{align*}
	[P(n+1) \land \forall m (P(m) \implies P(m\text{++}))] \implies& P(n+1) \\
	\implies& S(0) \;\; \text{(base)} \\
	\implies& \text{Assume $S(k)$ for some $k \geq 0$} \\
	\implies& \forall m P(m) \implies P(m\text{++}) \;\; \text{(given)} \\
	\implies& [P(n+1+k) \implies P((n+1+k) + 1)] \\
	\implies& [S(k) \implies S(k+1)] \\
	\implies& S(k) \forall k \geq 0 \;\; \text{(ordinary induction on $k$)} \\
	\implies& P(n+1+k) \forall k \geq 0\\
	\implies& P(m) \forall m \geq n+1 \\
	\end{align*}
 	}	
\end{enumerate}


\section{Repeated Addition (Multiplication)}

\begin{enumerate}[leftmargin=*, label=\textbf{\arabic*.}]
    \problem{Define the multiplication operation on two natural numbers, $n$ and $m$. \\ \\ We will now define multiplication as the process of repeated addition, analogously to how we defined addition as repeated incrementation. We define $0 \times m := 0$. Now suppose inductively that we define how to multiply $n$ to $m$. Then we can multiply $n\text{++}$ to $m$ by defining $$(n\text{++}) \times m := (n \times m) + m$$ \\
    Example: we can correctly multiply $3 \times 4$ by using nothing but this recursive definition of multiplication as repeated addition:
    \begin{align*}
    3 \times 4 &= (2\text{++}) \times 4 \\ 
                   &= (2 \times 4) + 4 \\
                   &= ((1\text{++}) \times 4) + 4 \\
                   &= ((1 \times 4) + 4) + 4 \\
                   &= (((0\text{++}) \times 4) + 4) + 4 \\
                   &= (((0 \times 4) + 4) + 4) + 4 \\  
                   &= (((0) + 4) + 4) + 4 \\
                   &= ((4) + 4) + 4 \\                              
                   &= (8) + 4 \\
                   &= 12 \\
    \end{align*}}
    
    \problem{(\textbf{EXERCISE 8}) Prove that multiplication is commutative.}
    \textcolor{blue}{Let $n$ and $m$ be natural numbers.}
   \end{enumerate}

\end{document}








